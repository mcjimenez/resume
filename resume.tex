\documentclass[]{resume}
\usepackage{fancyhdr}

\pagestyle{fancy}
\fancyhf{}

% Footers are not working possibly due to the minipage
% \rfoot{Page \thepage \hspace{1pt}}
% \lfoot{Latest version at \href{https://mcjimenez.github.io/resume}{https://mcjimenez.github.io/resume}}

\begin{document}


% Last update date
\lastupdated

% Name
\namesection{Carmen Jiménez Cabezas}{}{ Senior Software Engineer \\
\href{mailto:macajc@gmail.com}{{{\underline {macajc@gmail.com}}}} - TF: +34 682 681 246 \\
{\small Latest version at \href{https://mcjimenez.github.io/resume}{{{\underline {https://mcjimenez.github.io/resume}}}}}}

% Left column
\begin{minipage}[t]{0.66\textwidth}

% Experience

\section{Experience}
\runsubsection{Telefonica CDO}
\descript{| Senior Software Engineer }
\location{April 2018 - Current}
% \vspace{\topsep} % Hacky fix for awkward extra vertical space
\textbf{Role}: Senior Software Engineer.\\
\textbf{What I did}: Development of Android modules and libraries both in Java and Kotlin. Developing both the user fronting functionalities (UX) as well as the supporting functionality and some integration tools.\\
\textbf{Keywords}: Android, Java, Kotlin, ReactX, Dagger...
\sectionsep

\runsubsection{TokBox - CCDO}
\descript{| Profesional Services Engineering}
\location{Jul 2015 - Current}
% \vspace{\topsep} % Hacky fix for awkward extra vertical space
\textbf{Role}: Senior Software Engineer.\\
\textbf{What I did}: Development of ad hoc solutions for TokBox customers, from requisite capture to pre-production deployment, ensuring the highest possible quality for the developed code. Change management to keep developmet on time and budget.\\
Development both for backend (NodeJS, PHP) as well as frontend (Web, iOS, Windows .NET with WPF)\\
\textbf{Keywords}: C\#, JavaScript, HTML5, CSS, ObjectiveC, Electron, OpenTok, Swift...
\sectionsep

\runsubsection{Telefonica R\&D}
\descript{| Mozilla - FirefoxOS}
\location{Jun 2013 - July 2015}
% \vspace{\topsep} % Hacky fix for awkward extra vertical space
\textbf{Role}: Senior Security Software Engineer.\\
\textbf{What I did}: Building Firefox OS. Hacking on the Mozilla platform (AKA Gecko), Apps's WebAPI and permissions, Operator Variant, Vertical Homescreen (frontend and backend), Hello client for FirefoxOS and Desktop.\\
\textbf{Keywords}: C++, JavaScript, HTML5, CSS, shell script, Security.
\sectionsep

\runsubsection{Telefonica R\&D}
\descript{| Mozilla - FirefoxOS}
\location{Jun 2012 - Jun 2013}
\textbf{Role}: Security consultant.\\
\textbf{What I did}: In charge of project code's security audits. Discovering (and sometimes fixing) vulnerabilites on the Firefox/FirefoxOS code.\\
\textbf{Keywords}: C++, JavaScript, Fortify, Security.
\sectionsep

\runsubsection{Telefonica R\&D}
\descript{| Collaborative Security}
\location{Oct 2011 – May 2012}
\textbf{Role}: Software Engineer.\\
\textbf{What I did}: Saqqara (Saqqara-SC) Colaborative module development and quality assurance and integration with Saqqara-SA.\\
\textbf{Keywords}: J2EE, ZeroMQ, MySQL, SIEM, OSSIM, Security.
\sectionsep

\runsubsection{Telefonica Spain}
\descript{| Security in Services}
\location{Oct 2009 – Jan 2011}
\textbf{Role}: Security consultant.\\
\textbf{What I did}: Consulting on the implementation of security architectures for Telefonica Spain's Services. Defining, develop and validation of Security Operating Procedures.\\
\textbf{Keywords}: Solaris, Linux, Windows, Apache, Tomcat, JBoss, DBMS Oracle, MySql, Oracle Weblogic Server,...
\sectionsep

\end{minipage}
\hfill
\begin{minipage}[t]{0.33\textwidth}

% Skills

\section{Skills}
\subsection{Programming}
\location{{{\bf Proficiency}}}
J2EE \textbullet{} Java \textbullet{} JavaScript \textbullet{} ObjectiveC \\
\location{{{\bf Familiarity}}}
Kotlin \textbullet{} C \textbullet{} C++ \textbullet{} C\# \textbullet{} Swift \\
\sectionsep
\subsection{Worked/played with}
Android \textbullet{} iOS \\
PKI \textbullet{} digital signature \\
cipher algorithms and protocols \\
Linux \textbullet{} Unix \textbullet{} ShellScript \\
git \textbullet{} Peer review \textbullet{} Agile
\sectionsep

% Education
\section{Education}
\subsection{Polytechnic University}
\subsection{Of Madrid}
\descript{Degree on Computer Science}
\location{1999 | median mark 7,64}
\sectionsep

\section{Awards}
Undergraduate Thesis with honors:\\
'Diseño e implementación de un Sistema de Votación Electrónica Remota'\\
\sectionsep


% Public Software
\section{Public Software}
\textbullet{} iOS OpenTokApp \href{https://itunes.apple.com/us/app/opentokapp/id1106222570?mt=8}{{\underline{(link)}}}:
OpenTok demo application for iOS. My first iOS application, developed by myself from UX designs.


\sectionsep
\textbullet{} OpenTokDemo Application \href{https://opentokdemo.tokbox.com}{{\underline{(link)}}}:
OpenTok Demo Web Application.\\
First application I worked on as part of the TokBox PSE team. Developed by a team of 3 developers in 2 months.
\href{https://github.com/opentok/OpenTokRTC-V2}{{\underline{Source code link}}}\\

\sectionsep
\textbullet{} Chart Server\\
Standalone server to generate charts based on data sets. Can be customized with (\href{https://chartgeneratorprivate.herokuapp.com}{{\underline{here}}})
or without (\href{https://chartgenerator.herokuapp.com}{{\underline{here}}}) access restrictions \\
Developed just for fun\\
\href{https://github.com/mcjimenez/chartServer}{{\underline{Source code link}}}\\
\sectionsep

\end{minipage}
\hfill

\newpage

% Last update date
\lastupdated

% Name
\namesection{Carmen Jiménez Cabezas}{}{ Senior Software Engineer \\
\href{mailto:macajc@gmail.com}{macajc@gmail.com} - TF: +34 682 681 246}

% Left column
\begin{minipage}[t]{0.66\textwidth}

\runsubsection{Telefonica R\&D}
\descript{| Network and Services Security}
\location{Apr 2007 – Jun 2009}
\textbf{Role}: Software Engineer.\\
\textbf{What I did}: Training ATOS ORIGIN group, LDAP for Applications training for Security Strategy and Development Network group. ATICO Development, Infradha develpment.\\
\textbf{Keywords}: HTML, JavaScript, SunOne Application Server, SunOne Directory Server, Active Directory, Radius, DBMS Oracle, TACACS+.
\sectionsep

\runsubsection{Lucent Technologies}
\descript{|}
\location{Ago 2006 – Apr 2007}
\textbf{Role}: Software Engineer.\\
\textbf{What I did}: Peer review of drivers and self-configuration software for hardware devices and extending the system to allow distributed network configuration for the hardware devices.\\
\textbf{Keywords}: J2EE, C, DBMS Oracle.
\sectionsep

\runsubsection{Telefonica R\&D}
\descript{| Security Management and Access Control}
\location{Jan 2006 – Ago 2006}
\textbf{Role}: Software Engineer.\\
\textbf{What I did}: Assess the maturity and viability of different authentication and identity management technologies to achieve a unique identity for Telefonica customers on all our services.\\
\textbf{Keywords}: Liberty Alliance, HTML, JavaScript, J2EE, SunOne Application Server, Identity.
\sectionsep

\runsubsection{Telefonica R\&D}
\descript{| Cryptography and Digital Identity Services}
\location{Apr 2000 – Jun 2006}
\textbf{Role}: Software Engineer.\\
\textbf{What I did}: Single Sign On based on a certificate stored in a smart card. Digital signature API.\\
\textbf{Keywords}: PKI, J2EE, SunOne WebService, DBMS Oracle, C++, ActiveX, NSPR, NSS, Identity.
\sectionsep

\runsubsection{Airtel (Vodafone)}
\descript{| Comisiones project}
\location{Dec 1999 – Apr 2000}
\textbf{Role}: Junior Software Engineer.\\
\textbf{What I did}: Business logic in the application.\\
\textbf{Keywords}: Visual Basic, DBMS Oracle.
\sectionsep

\end{minipage}
\hfill
\begin{minipage}[t]{0.33\textwidth}

% Last Presentation
\section{Last Presentation}
\textbullet{} Peer Reviews (\href{http://mcjimenez.github.io/presentations/peerReviewWoW}{{\underline{link}}})\\
Last public presentation for coworkers from other teams in the company\\
\textbf{Objective}: Learn how peer review works and why use it, git.
\sectionsep

% Open Source
\section{Open Source}
\begin{tabular}{rll}
Firefox Browser & Contributor\\
Firefox OS & Contributor\\
PassportJS & \href{https://www.npmjs.com/package/passport-yahoo-new-oauth2}{{\underline{Yahoo OAuth2}}}\\
\end{tabular}
\sectionsep

% Pesonal skills

\section{Personal skills}
Easy going \textbullet{} Team player \\
Loves helping people to improve \\
Proactive personality \textbullet{} Avid learner \\
If it can be done I can learn it
\sectionsep

\section{Motto}
Done is better than perfect, but the goal is perfection.
\sectionsep

% Interests
\section{Interests}
Open Source Software Engineering \\
The Web, Web APIs, Web Services \\
Cloud Computing, Distributed Systems, \\
Security and Cryptography
\sectionsep

\end{minipage}

\end{document}
